\documentclass{article}
\usepackage{multicol}
\usepackage[utf8]{inputenc}
\usepackage[textwidth=460pt, voffset=0pt]{geometry}
\usepackage{fancyhdr}
\usepackage{graphicx}
\usepackage{ulem}
\begin{document}
\title{\vspace{-5ex}Nadir}
\author{\vspace{-5ex}}
\date{\vspace{-5ex}}
\pagestyle{fancy}
\fancyhf{}
\lhead{ACIO 2023 Contest 2.5}
\rfoot{Page \thepage}

\begin{center}
\huge{Nadir}\small\\
\vspace{5ex}
\begin{tabular}{|c|c|} 
\hline
Time Limit & Memory Limit \\
\hline
1 second & 128 MB \\

\hline
\end{tabular}
\end{center}
\section*{Statement}

There are $N$ towers of cubes in a row, numbered 1 to $N$ from left to right. The $i$th tower has $a_i$
cubes in it. A row of towers is a {\it valley} if it can be split into two (possibly empty) sections where:
\begin{itemize}
\item The left section forms a non-increasing sequence, and
\item The right section forms a non-decreasing sequence.
\end{itemize}

\sout{Unfortunately, due to the recent collapse of the cryptocurrency exchange FTX, we have lost our source of funding and cannot provide you with a diagram.}

You are tasked with considering $Q$ scenarios: In the $i$th scenario, what is the fewest cubes you must
add to turn the towers $1, \dots, k_i$ into a valley? You cannot remove any cubes.

\section*{Input}

\begin{itemize}
\item The first line of input contains the integers $N$ and $Q$.
\item The next line of input contains the space-separated integers $a_1, \dots, a_N$.
\item The next line of input contains the space-separated integers $k_1, \dots, k_Q$.
\end{itemize}

\section*{Output}

Output $Q$ lines, the $i$th of which is the answer for the $i$th scenario.

\begin{multicols}{2}
\section*{Sample Input 1}
{\tt
6 3\\
4 2 3 1 6 6\\
3 4 6
}
\columnbreak
\section*{Sample Input 2}
{\tt
5 3\\
1 2 3 2 1\\
1 3 5
}
\end{multicols}
\begin{multicols}{2}
\section*{Sample Output 1}
{\tt
0\\
1\\
1
}
\columnbreak
\section*{Sample Output 2}
{\tt
0\\
0\\
3
}
\end{multicols}

\section*{Constraints}
\begin{itemize}
\item $1 \le N, Q \le 10^5$
\item $1 \le a_i \le 10^9$
\item $1 \le k_j \le N$
\item All $k_j$ are unique and in ascending order.
\end{itemize}

\section*{Subtasks}
\begin{tabular}{l*{6}{c}r}
Number & Points & Additional constraints\\
\hline
1 & 25 & $Q = 1$ and $k_j = N$ \\
2 & 26 & $a_i \le 10$ \\
4 & 49 & No further constraints
\end{tabular}
\end{document}
